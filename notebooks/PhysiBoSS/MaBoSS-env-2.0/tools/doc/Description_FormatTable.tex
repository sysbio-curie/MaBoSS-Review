\documentclass{article}
\usepackage{amsmath, amsthm, amssymb,algorithm2e}
\newcommand{\prob}{\mathbf{P}}

\begin{document}
\section*{MBSS\_FormatTable.pl}
\subsection*{Aim}
FormatMaBoSS.pl is a perl script that facilitates the use of MaBoSS, by creating tables from output files in a dedicated folder.
\subsection*{How to run the script}
It is run as a command line; inputs files need to be provided; MaBoSS executable name and threshold are optional.
\begin{verbatim}
MBSS_FormatTable.pl <file.bnd> <file.cfg> <(optional)threshold> 
  <(optional)-mb MaBoSS executable name>
\end{verbatim}

Files should be in working directory or accessible by absolute path.
MaBoSS executable should be in working directory, accessible by absolute path or by environment variable. 
A threshold is used for removing low probability in the table of stationary distribution decomposition. If no threshold is provided, the file ``file\_statdist\_table.cfg'' is not constructed.

\subsection*{Outputs}
A folder is created, with the same name as the cfg file (adding a number as suffix if directory already exists). MaBoSS executable, bnd and cfg files are copied in this new folder (and MaBoSS executable name if it is a readable file).
MaBoSS is run in the new folder, with the given bnd and cfg files as inputs. MBSS\_FormatTable.pl creates tables (in text format) from output files, in the new folder: ``file\_probtraj\_table.csv'' (from ``file\_probtraj.csv'')  and ``file\_statdist\_table.cfg'' (``from file\_statdist.cfg'' if the threshold is given). These tables have a column for each network state. These tables can be easily downloaded in a spreadsheet.

 NB: these table are less compact than the output files produced directly by MaBoSS. These could be large, if the set of network states is large.

\end{document}
